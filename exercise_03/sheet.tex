% kompiliert mit XeLaTeX
\documentclass{scrartcl}

% Deutsch, fontspec, microtype
\usepackage {polyglossia}
\setmainlanguage {english}
\usepackage {xltxtra}
\usepackage{microtype}

% Package für Bilder
\usepackage{graphicx}

% Mathe
\usepackage{amsmath}

% Kopfzeile
\usepackage{scrlayer-scrpage}
\pagestyle {scrheadings}
\rohead*{Kevin Lange\\Max Simon}

\begin{document}
	
	\part*{Exercise Sheet 03}	
	\section*{Part 2}
	For a better understanding of the algorithm, we implemented it in Python (see \emph{tree_alg.py}). The code is not performant nor optimized for memory consumption, but gives opportunities to check the workflow. For example one can use the method \emph{proof_all_particles} to proof, if all particles are in the right node, can print out the whole tree (with \emph{print_tree}) and has some informational output. The code has a good documentation, so we don't want to explain it here.\\
	\\
	We inserted N = 5000, 10000, 20000 and 40000 particles (uniformly distributed in the box with a mass of \frac{1}{N}) for different opening angles thresholds ($\Theta = 0.2, 0.4 \text{and} 0.8$). This is done by the static method \emph{init_a_tree}.\\
	\\
	The \emph{analyzing} method do an exact calculation and an calculation using the tree method. Afterwards it prints out the used time for both calculations, the mean value of $\eta$ and the mean number of nodes used for the calculation of one particle.\\
	\\
	In the \emph{part2.py} script we initialize and start the simulations. Afterwards the measured times and the mean $\eta$ depending on N are plotted for each $\theta$. The results can be found in %TODO!!! 
	
	
\end{document}